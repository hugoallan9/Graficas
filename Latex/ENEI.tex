% !TEX encoding = UTF-8 Unicode
\documentclass[12pt,letterpaper,twoside]{book}

%\usepackage[utf8]{inputenc}  %Para compilar en PdfLaTeX


%Recordatorio de relleno

	%macro del capítulo
	%\INEchapter{TÍTULO: SUBTÍTULO (para índice)}{TÍTULO:}{SUBTÍTULO}{Descripción}
	%Hoja
	
	%caja de media hoja
	%\cajita{Título}{Descripción}{Subtítulo}{Desagregación}{Gráfica}{Fuente}{\notita{Nota}}
	
	%caja de hoja completa
	%\cajota{Título}{Descripción}{Subtítulo}{Desagregación}{Gráfica}{Fuente}{\notita{Nota}}



%Paquetes estándar
\usepackage{amsmath}
\usepackage{amsfonts}
\usepackage{amssymb}
\usepackage{graphicx}
\usepackage{pdfpages}
\usepackage{setspace} 
\usepackage{xltxtra}


%Tablas de Excel convertidas a LaTeX
\usepackage{booktabs}
\usepackage{multirow}
\newcounter{Cuadro}[chapter]
\renewcommand{\theCuadro}{\thechapter.\arabic{Cuadro}}


%Columnas definibles en ancho
\usepackage{array}
	\newcolumntype{x}[1]{%
	>{\centering\arraybackslash}p{#1}}%
	
	\newcolumntype{g}[1]{%
	>{\raggedleft\arraybackslash}p{#1}}%


\usepackage[input-decimal-markers={.}, input-ignore={,}, group-separator={,}]{siunitx}


%Para pruebas
\usepackage{lipsum}
\newcommand{\comop}{}
\newcommand{\comcl}{}
\newcommand{\guiop}{}
\newcommand{\guicl}{}
\newcommand{\apartado}[1]{\addtocounter{section}{1}
{\noindent\Bold\Large\color{color1!80!black}\thesection $\,-$ #1}\\[1mm]}


%Para compilar en XeLaTeX con tildes
\usepackage{polyglossia}
	\setmainlanguage{spanish}

%Tipo de letra
\usepackage{fontspec}
	\setmainfont[
		BoldFont = OpenSans-CondBold.ttf ,
		ItalicFont = texgyrepagella-italic.otf ,
		BoldItalicFont = OpenSans-CondLightItalic.ttf ]{OpenSans-CondLight.ttf}
	\newfontfamily\Bold{Open Sans Condensed Bold}
	
	\newfontfamily\Sans{Open Sans}
	\newfontfamily\Italic{Open Sans Condensed Light Italic}
	\newfontfamily\Logos{Latin Modern Roman}
	


%Diseño global del documento
\usepackage[width=217mm, height=279mm, left=22.60mm, right=13.56mm, top=26.43mm, bottom=27.86mm]{geometry}
	\setlength{\headsep}{0pt}
	\setlength{\footskip}{46pt}
	\setlength{\parindent}{2em}		%sangría
	\setlength{\parskip}{2ex}		%separación entre párrafos  
	
\newcommand{\swapgeometry}{\clearpage
\newgeometry{left=28.60mm, right=21.56mm, top=24.43mm, bottom=27.86mm}
\setlength{\footskip}{26pt}}

	%Distancias
		\newlength{\distx} 
		\setlength{\distx}{4.48mm}
		\newlength{\disty} 
		\setlength{\disty}{5.75mm}



%Tabla de contenidos y vinculaciones
\usepackage{tocloft}
\usepackage[hidelinks]{hyperref}
\usepackage{url}

	%Formato de tabla de contenidos
	\setlength{\cftbeforetoctitleskip}{0em}
	\AtBeginDocument{\addtocontents{toc}{\protect\thispagestyle{empty}}} 

	\makeatletter
		\renewcommand*\l@subsection{\@dottedtocline{2}{5.2em}{3.2em}}
	\makeatother
	
	\renewcommand{\thesection}{\thechapter.\arabic{section}}
	
	\cftsetpnumwidth{2\distx}
	\cftsetrmarg{8\distx}
	\renewcommand{\cftsecnumwidth}{2.0\distx}
	\renewcommand{\cftchapnumwidth}{2\distx}
	\renewcommand{\cftsecindent}{2\distx}



% Elementos geométricos de diseño del cuerpo (cajas de colores, etc.)
\usepackage{colortbl}


%\usepackage[usenames,dvipsnames,svgnames,table]{xcolor}

	%Cambios de márgenes y según paridad de hojas
		\usepackage[strict]{changepage}
			\strictpagecheck

	%Colores base del documento
		\definecolor{color1}{rgb}{0.82,0.49,0.1568}
		\definecolor{color2}{rgb}{0.48235,0.607843,0.219607843}


	%Para que las páginas en blanco no estén numeradas
		\let\origdoublepage\cleardoublepage
		\newcommand{\clearemptydoublepage}{
  		\clearpage
  		{\pagestyle{empty}\origdoublepage}}
		\let\cleardoublepage\clearemptydoublepage
		
	%Llamadas hacia notas
		\newcommand{\llamada}{*}
		
	%tablas
		\setlength{\arrayrulewidth}{1.0pt}
		\arrayrulecolor{color1!80!black}
		
		%Columnas centradas definibles en ancho
		\newcolumntype{x}[1]{%
		>{\centering\arraybackslash}p{#1}}%



% Tcolorbox
\usepackage[skins, breakable, hooks]{tcolorbox}


\newtcolorbox{tocbox}{skin=enhancedmiddle, width=39\distx, nobeforeafter, boxrule=0pt, colframe=white, left=0\distx, enlarge left by = 4\distx, enlarge right by=2\distx, bottom=0pt, top=0pt, right=0\distx, left=0\distx, arc=0pt, colback= white, breakable,check odd page,toggle left and right}

\newtcolorbox{fondo}{width=40\distx, height=39\disty, nobeforeafter, boxrule=0pt, colframe=white, left=0pt, bottom=0pt, top=0pt, right=0pt, arc=0pt, colback= white}

\newtcolorbox{bloque-media}{width=40\distx, height=19\disty, enlarge top by=-3pt, enlarge left by=-3pt, enlarge bottom by=1.4\disty, nobeforeafter, colframe=white, colback=white, left=0pt, right=0pt, bottom= 0pt, top=0pt, arc=0pt, boxrule=0pt}

\newtcolorbox{bloque-una}{width=40\distx, height=1\disty, enlarge top by=-3pt, enlarge left by=-3pt, enlarge bottom by=1.4\disty, nobeforeafter, colframe=white, colback=white, left=0pt, right=0pt, bottom= 0pt, top=0pt, arc=0pt, boxrule=0pt}

\newtcolorbox{descripcion}{width=10.5\distx, height=15.9\disty, enlarge bottom by=0\disty, enlarge top by=-3pt, enlarge left by=-3pt, nobeforeafter, boxrule=3pt, colback=color1!8!white, colframe=color1!8!white, left=5pt, right=5pt, top=4pt,bottom=4pt}

\newtcolorbox{descripcion-una}{width=33.3\distx, height=6.2\disty, enlarge bottom by=1\disty, enlarge top by=-5pt, enlarge left by=-3pt, nobeforeafter, boxrule=3pt, colback=color1!8!white, colframe=color1!8!white, left=5pt, right=5pt, top=3pt,bottom=4pt}

\newtcolorbox{notita-impar}{width=5.9\distx, enlarge bottom by=0\disty, enlarge top by=1\disty, enlarge left by=0.1\distx, nobeforeafter, boxrule=0pt, colback=white, colframe=color2!40!white, left=3pt, right=1pt, top=2pt,bottom=2pt, leftrule=0.6pt, rightrule=0pt,toprule=0pt,bottomrule=0pt, arc=0pt}

\newtcolorbox{notita-par}{width=5.9\distx, enlarge bottom by=0\disty, enlarge top by=1\disty, enlarge left by=-0.55\distx, nobeforeafter, boxrule=0pt, colback=white, colframe=color2!40!white, left=1pt, right=3pt, top=2pt,bottom=2pt, rightrule=0.6pt, leftrule=0pt,toprule=0pt,bottomrule=0pt, arc=0pt}

\newtcolorbox{numero-subseccion}{width=2.6\distx, height=1.4\disty, enlarge top by=-3pt, enlarge left by=-2.8\distx, enlarge right by= 4.613\distx, nobeforeafter, boxrule=0pt, colback=white, colframe=white, left=0.1\distx, right=0.1\distx, bottom= 0.455\distx, top=0.145\distx, arc=0pt}

\newtcolorbox{titulo-subseccion}{width=28.8\distx, enlarge top by=-3pt, enlarge left by=-0.2\distx, enlarge bottom by=0\distx, nobeforeafter, boxrule=0pt, colback=white, colframe=white, left=0.1\distx, right=0.1\distx, bottom= -0.4\distx, top=0\distx, arc=0pt}

\newtcolorbox{titulo-subseccion-blanco}{width=28\distx, enlarge top by=-0pt, enlarge left by=0pt, enlarge bottom by=0pt, nobeforeafter, colback=white, colframe=white, left=0\distx, right=0.5\distx, bottom= -4pt, top=-2pt, arc=0pt, bottomrule=0pt, leftrule=0pt, toprule= 0mm, rightrule= 0mm}

\newtcolorbox{titulo-subseccion-continuacion}{width=32\distx, enlarge top by=-3pt, enlarge left by=-0.2\distx, enlarge bottom by=0.4\distx, nobeforeafter, boxrule=0pt, colback=white, colframe=white, left=0.3\distx, right=0.1\distx, bottom= -0.4\distx, top=0\distx, arc=0pt}

\newtcolorbox{titulo}{width=34\distx, height=3\disty, enlarge top by=-3pt, enlarge left by=-3pt, enlarge bottom by=0.2\disty, nobeforeafter, colback=white, colframe=white, left=0.5\distx, right=0.5\distx, top=32pt,bottom=-48pt, arc=0pt, boxrule=0pt}

\newtcolorbox{centrador}{width=34\distx, enlarge top by=-48pt, enlarge left by=-3pt, enlarge bottom by=0pt, nobeforeafter, colback=white, colframe=white, left=-3pt, right=-3pt, bottom= -3pt, top=-3pt, arc=0pt, boxrule=0pt}

\newtcolorbox{centrador-par}{width=34\distx, enlarge top by=-48pt, enlarge left by=66.7pt, enlarge bottom by=0pt, nobeforeafter, colback=white, colframe=white, left=-3pt, right=-3pt, bottom= -3pt, top=-3pt, arc=0pt, boxrule=0pt}

\newtcolorbox{subtitulo}{width=22\distx, height=3\disty, enlarge top by=-3pt, enlarge left by=-3pt, enlarge bottom by=0.1\disty, nobeforeafter, colframe=white, colback=white, left=0\distx, right=0\distx, bottom= 0pt, top=0pt, arc=0pt, boxrule=0pt}

\newtcolorbox{subtitulo-una}{width=28\distx, height=3\disty, enlarge top by=-3pt, enlarge left by=-3pt, enlarge bottom by=0.1\disty, nobeforeafter, colframe=white, colback=white, left=0\distx, right=0\distx, bottom= 0pt, top=0pt, arc=0pt, boxrule=0pt}

\newtcolorbox{grafica}{width=22.1\distx, height=12\disty, enlarge top by=-3pt, enlarge left by=-5pt, enlarge bottom by=0.1\disty, nobeforeafter, colframe=white, colback=white, left=-2pt, right=-2pt, bottom= -2pt, top=-4pt, arc=0pt, boxrule=0pt}

\newtcolorbox{grafica-una}{width=33.4\distx, height=24.9\disty, enlarge top by=-3pt, enlarge left by=-5pt, enlarge bottom by=0.1\disty, nobeforeafter, colframe=white, colback=white, left=-2pt, right=-2pt, bottom= -2pt, top=-4pt, arc=0pt, boxrule=0pt}

\newtcolorbox{fuente}{width=22\distx, height=1\disty, enlarge top by=-3pt, enlarge left by=-3pt, enlarge bottom by=0\disty, nobeforeafter, colframe=white, colback=white, left=0pt, right=0pt, bottom= 0pt, top=2pt, arc=0pt, boxrule=0pt}

\newtcolorbox{fuente-una}{width=33.4\distx, height=1\disty, enlarge top by=-3pt, enlarge left by=-3pt, enlarge bottom by=0\disty, nobeforeafter, colframe=white, colback=white, left=0pt, right=0pt, bottom= 0pt, top=2pt, arc=0pt, boxrule=0pt}

\newtcolorbox{columna-central}{width=22\distx, height=15.9\disty, enlarge top by=-3pt, enlarge left by=-8pt, enlarge bottom by=0\disty, nobeforeafter, colframe=white, colback=white, left=0pt, right=0pt, bottom= 0pt, top=0pt, arc=0pt, boxrule=0pt}

\newtcolorbox{columna-central-una}{width=33.4\distx, height=28.9\disty, enlarge top by=-3pt, enlarge left by=-4pt, enlarge bottom by=0\disty, nobeforeafter, colframe=white, colback=white, left=0pt, right=0pt, bottom= 0pt, top=0pt, arc=0pt, boxrule=0pt}

\newtcolorbox{vacio1}{width=6\distx, height=3\disty, enlarge top by=-3pt, enlarge left by=-3pt, enlarge bottom by=0.1\disty, nobeforeafter, colframe=white, colback=white, left=0pt, right=0pt, bottom= 0pt, top=0pt, arc=0pt, boxrule=0pt}

\newtcolorbox{nota}{width=6\distx, height=12\disty, enlarge top by=-3pt, enlarge left by=-3pt, enlarge bottom by=0.1\disty, nobeforeafter, colframe=white, colback=white, left=0pt, right=0pt, bottom= 0pt, top=0pt, arc=0pt, boxrule=0pt}

\newtcolorbox{nota-una}{width=6\distx, height=24.9\disty, enlarge top by=-3pt, enlarge left by=-3pt, enlarge bottom by=0.1\disty, nobeforeafter, colframe=white, colback=white, left=0pt, right=0pt, bottom= 0pt, top=0pt, arc=0pt, boxrule=0pt}

\newtcolorbox{vacio2}{width=6\distx, height=1\disty, enlarge top by=-3pt, enlarge left by=-3pt, enlarge bottom by=0\disty, nobeforeafter, colframe=white, colback=white, left=0pt, right=0pt, bottom= 0pt, top=0pt, arc=0pt, boxrule=0pt}

\newtcolorbox{columna-marginal}{width=6\distx, height=15.9\disty, enlarge top by=-3pt, enlarge left by=-8pt, enlarge bottom by=0\disty, nobeforeafter, colframe=white, colback=white, left=0pt, right=0pt, bottom= 0pt, top=0pt, arc=0pt, boxrule=0pt}

\newtcolorbox{columna-marginal-una}{width=6\distx, height=28.9\disty, enlarge top by=-3pt, enlarge left by=-8pt, enlarge bottom by=0\disty, nobeforeafter, colframe=white, colback=white, left=0pt, right=0pt, bottom= 0pt, top=0pt, arc=0pt, boxrule=0pt}


% Encabezado y pie de página
\usepackage{fancyhdr}

	%cajitas de encabezado y pie de página
	\newtcbox{pagina}{nobeforeafter, boxrule=0.6pt, colback=color2!98!white, colframe=color2!98!white, left=3pt, right=3pt, bottom= 2pt, top=3pt, arc=0pt, enlarge left by=-5pt, enlarge right by=-5pt}
	
	\newtcbox{piecapitulo}{nobeforeafter, boxrule=0pt, colback=white, colframe=color2!98!white, bottomrule=0.6pt, left=0pt, right=0pt, bottom= -3.45pt, top=3pt, arc=0pt}

	%definición de estilo estándar de página
	\fancypagestyle{estandar}{%
	\fancyhf{}
	\fancyfoot[RO]{\setlength{\arrayrulewidth}{0.7pt}\arrayrulecolor{color2!95!black}\color{color2!95!black}
		\begin{tabular}{p{10cm}|x{0.5cm}}
	 &\\[-0.4cm]
	\hfill\chaptitle$\,\,$ &    \thepage  \\ 
	\end{tabular} \hspace{-1.2cm} \arrayrulecolor{color1!80!black}
	}
	\fancyfoot[LE]{\setlength{\arrayrulewidth}{0.7pt}\arrayrulecolor{color2!95!black}\color{color2!95!black}
	\hspace{-1.2cm}\begin{tabular}{x{0.5cm}|p{10cm}}
	&\\[-0.4cm]
	 \thepage &  $\,\,$Encuesta Nacional de Empleos e Ingresos 1-2014 \\ 
	\end{tabular} \arrayrulecolor{color1!80!black}
	}
	\renewcommand{\headrulewidth}{0pt}
	\renewcommand{\footrulewidth}{0pt}}

\pagestyle{empty}

%Formato de encabezados
	\usepackage[explicit]{titlesec}
	%\usepackage{sectsty}


	\titleformat{\subsection}[runin]{}{}{0pt}{}[]
	\titlespacing{\subsection}{0pt}{-20pt}{0pt}
	
	\titleformat{\section}[runin]{}{}{0pt}{}[]
	\titlespacing{\section}{0pt}{-20pt}{0pt}
		
	\titleformat{\chapter}[runin]{}{}{0pt}{}[]
	\titlespacing{\section}{0pt}{-20pt}{0pt}
		
	%Comandos recolectores de información del pie de página
		\newcommand{\chaptitle}{}
		\newcommand{\subsectitle}{}
		\newcommand{\sectitle}{}
	 	 
	%Cajitas de capítulo
		\newtcolorbox{numero-capitulo}{width=4.5\distx, height= 5\disty, enlarge top by=-3pt, enlarge left by=-3pt, nobeforeafter, boxrule=0pt, colback=color1!95!black, colframe=color1!95!black, left=0\distx, right=0\distx, bottom= 0.2\distx, top=1.2\distx, arc=0pt}	
		
		\newtcolorbox{titulo-capitulo}{enhanced,width=34\distx, height= 5\disty, enlarge top by=-3pt, enlarge left by=1.5cm, nobeforeafter, boxrule=0pt, interior style={left color=color1!10!white,
		right color=white}, left=0\distx, right=-4.5\distx, bottom= 0\distx, top=0\distx, arc=0pt}
		
		\newtcolorbox{numero-capitulo-long}{width=6\distx, height= 5\disty, enlarge top by=-3pt, enlarge left by=-3pt, nobeforeafter, boxrule=0pt, colback=color1!95!black, colframe=color1!95!black, left=0\distx, right=0\distx, bottom= 0.2\distx, top=1.2\distx, arc=0pt}	
		
		\newtcolorbox{titulo-capitulo-long}{enhanced,width=34\distx, height= 5\disty, enlarge top by=-3pt, enlarge left by=1.5cm, nobeforeafter, boxrule=0pt, interior style={left color=color1!10!white,
		right color=white}, left=0\distx, right=-6\distx, bottom= 0\distx, top=0\distx, arc=0pt}
		
		\newtcolorbox{capitulo-descripcion}{width=22\distx, enlarge top by=-3pt, enlarge left by=-3pt, nobeforeafter, enlarge right by = 6\distx, colback=white, colframe=color1!20!white, left=1.5\distx, right=2\distx, bottom= 1\distx, top= 1\distx, arc=0pt, bottomrule=0pt, leftrule=2pt, toprule= 0pt, rightrule= 0pt}
		
		
	%Cajitas de apéndice
		\newtcolorbox{numero-capitulo-app}{width=4.5\distx, height= 5\disty, enlarge top by=-3pt, enlarge left by=-3pt, nobeforeafter, boxrule=0pt, colback=color2!95!black, colframe=color2!95!black, left=0\distx, right=0\distx, bottom= 0.2\distx, top=1.2\distx, arc=0pt}	
		
		\newtcolorbox{titulo-capitulo-app}{enhanced,width=34\distx, height= 5\disty, enlarge top by=-3pt, enlarge left by=1.5cm, nobeforeafter, boxrule=0pt, interior style={left color=color2!10!white,
		right color=white}, left=0\distx, right=-4\distx, bottom= 0\distx, top=0\distx, arc=0pt}
		
		\newtcolorbox{capitulo-descripcion-app}{width=22\distx, enlarge top by=-3pt, enlarge left by=-3pt, nobeforeafter, enlarge right by = 6\distx, colback=white, colframe=color2!20!white, left=1.5\distx, right=2\distx, bottom= 1\distx, top= 1\distx, arc=0pt, bottomrule=0pt, leftrule=2pt, toprule= 0pt, rightrule= 0pt}


%macro de sección	
	 	
	\newcommand{\subsecnew}[1]{\renewcommand{\subsectitle}{#1} \subsection{#1} {\Bold\large \subsectitle\\[-0.35cm]}}

	\newcounter{secnumber}[chapter]
	
	\newcommand{\secnew}[1]{\stepcounter{secnumber}\renewcommand{\sectitle}{#1} \section{#1} { \raisebox{0pt}{\begin{titulo-subseccion-blanco} \Bold\large\sectitle\vphantom{p}\end{titulo-subseccion-blanco}}\\[-0.35cm]}}



% Formato de números de sección, subsección, etc...
	
	\newcommand{\secnumbering}{{\large\Bold \thechapter.\thesecnumber}}



%Macros de relleno de contenido

\newcommand{\INEchapter}[4]{\cleardoublepage\addtocontents{toc}{\protect\addvspace{0.6\baselineskip}\color{color1!80!black}}\chapter[\texorpdfstring{\color{color1!80!black}#1}{#1}]{#2}\renewcommand{\chaptitle}{#1}\thispagestyle{empty}\addtocontents{toc}{\protect\addvspace{0.3\baselineskip}{\color{color1!10!white}\hrule height 0.9pt} \addvspace{0.6\baselineskip} \color{black}} \stepcounter{secnumber} $\ $\\[-1cm]
	 \begin{titulo-capitulo}
	 	\begin{numero-capitulo}\centering
			{\fontsize{24mm}{1em}\selectfont\color{white} \Bold \thechapter}
		\end{numero-capitulo}\quad 
		\raisebox{2.1\disty}{\begin{tabular}{l}
		\fontsize{9.5mm}{1em}\selectfont \Bold \color{color1!95!black} #2\\ 
		\fontsize{9.5mm}{1em}\selectfont \Bold \color{color1!95!black} #3 \vphantom{Í} 
		\end{tabular}}
	\end{titulo-capitulo}
		$\ $\\[1.5cm]
		\begin{flushright}
			\begin{capitulo-descripcion}
				\large #4
			\end{capitulo-descripcion}
		\end{flushright}
		\cleardoublepage
		}
		

\newcommand{\INEchapterlong}[4]{\cleardoublepage\addtocontents{toc}{\protect\addvspace{0.6\baselineskip}\color{color1!80!black}}\chapter[\texorpdfstring{\color{color1!80!black}#1}{#1}]{#2}\renewcommand{\chaptitle}{#1}\thispagestyle{empty}\addtocontents{toc}{\protect\addvspace{0.3\baselineskip}{\color{color1!10!white}\hrule height 0.9pt} \addvspace{0.6\baselineskip} \color{black}} \stepcounter{secnumber} $\ $\\[-1cm]
	 \begin{titulo-capitulo-long}
	 	\begin{numero-capitulo-long}\centering
			{\fontsize{24mm}{1em}\selectfont\color{white} \Bold \thechapter}
		\end{numero-capitulo-long}\quad 
		\raisebox{2.1\disty}{\begin{tabular}{l}
		\fontsize{9.5mm}{1em}\selectfont \Bold \color{color1!95!black} #2\\ 
		\fontsize{9.5mm}{1em}\selectfont \Bold \color{color1!95!black} #3 \vphantom{Í} 
		\end{tabular}}
	\end{titulo-capitulo-long}
		$\ $\\[1.5cm]
		\begin{flushright}
			\begin{capitulo-descripcion}
				\large #4
			\end{capitulo-descripcion}
		\end{flushright}
		\cleardoublepage
		}
		
		
\newcommand{\appchapter}[4]{\cleardoublepage\addtocontents{toc}{\protect\addvspace{-0.1\baselineskip}\color{color2!80!black}}\chapter[\texorpdfstring{\color{color2!80!black}#1}{#1}]{#2}\renewcommand{\chaptitle}{#1}\thispagestyle{empty} \stepcounter{secnumber}\stepcounter{Cuadro} $\ $\\[1cm]
	 \begin{titulo-capitulo-app}
	 	\begin{numero-capitulo-app}\centering
			{\fontsize{24mm}{1em}\selectfont\color{white} \Bold \thechapter}
		\end{numero-capitulo-app}\quad 
		\raisebox{2.1\disty}{\begin{tabular}{l}
		\fontsize{9.5mm}{1em}\selectfont \Bold \color{color2!95!black} #2\\ 
		\fontsize{9.5mm}{1em}\selectfont \Bold \color{color2!95!black} #3 \vphantom{Í} 
		\end{tabular}}
	\end{titulo-capitulo-app}
		$\ $\\[1.5cm]

		\cleardoublepage
		}



\newcommand{\hoja}[1]{\noindent\begin{fondo} #1 \end{fondo}\clearpage}



\newcommand{\notita}[1]{\checkoddpage\ifoddpage
	\begin{notita-impar}
		\scriptsize\color{color2!98!white} #1
	\end{notita-impar}
\else
	\begin{notita-par}
		\scriptsize\color{color2!98!white} #1
	\end{notita-par}
\fi
}



\newcommand{\cajita}[7]{\checkoddpage\ifoddpage
\begin{bloque-media}
\begin{titulo}
\begin{centrador}
	\begin{tabular}{p{2.5\distx}p{28.8\distx}}
	 & \\[-3mm]	
	 & {\begin{titulo-subseccion}\begin{numero-subseccion}\secnumbering \end{numero-subseccion}  \phantomsection{\secnew{#1}}  \end{titulo-subseccion}}  \\[-5mm]
	 &  \\[-1.3pt]
	\end{tabular}
\end{centrador}
\end{titulo}

\begin{tabular}{p{11\distx}p{21\distx}p{5\distx}}
		\begin{descripcion}
			#2
		\end{descripcion} 
	& 
		\begin{columna-central}
			\begin{subtitulo}
				\centering\footnotesize{\Bold #3} \\
				$\qquad\,\,$#4$\qquad\,\,$
			\end{subtitulo}
	
			\begin{grafica}\centering
				#5
			\end{grafica}
	
			\begin{fuente}
				\footnotesize #6 
			\end{fuente}
		\end{columna-central}
	& 
		\begin{columna-marginal}
			\begin{vacio1}
			
			\end{vacio1}
		
			\begin{nota}
				#7
			\end{nota}
		
			\begin{vacio2}
			
			\end{vacio2}
		\end{columna-marginal}
	\\
\end{tabular}
\end{bloque-media}
\else
\begin{bloque-media}
\begin{titulo}
\begin{centrador-par}
	\begin{tabular}{p{2.5\distx}p{28.8\distx}}
	 & \\[-3mm]	
	 & {\begin{titulo-subseccion}\begin{numero-subseccion}\centering\secnumbering\end{numero-subseccion}  \phantomsection{\secnew{#1}}  \end{titulo-subseccion}}  \\[-5mm]
	 &  \\[-1.3pt]
	\end{tabular}
\end{centrador-par}
\end{titulo}

\begin{tabular}{p{5\distx}p{21.36\distx}p{11\distx}}
		\begin{columna-marginal}
			\begin{vacio1}
			
			\end{vacio1}
		
			\begin{nota}
				#7
			\end{nota}
		
			\begin{vacio2}
			
			\end{vacio2}
		\end{columna-marginal}		
	& 
		\begin{columna-central}
			\begin{subtitulo}
				\centering\footnotesize{\Bold #3} \\
				$\qquad\,\,$#4$\qquad\,\,$
			\end{subtitulo}
	
			\begin{grafica}\centering
				#5
			\end{grafica}
	
			\begin{fuente}
				\footnotesize #6 
			\end{fuente}
		\end{columna-central}
	& 
		\begin{descripcion}
			#2
		\end{descripcion} 
	\\
\end{tabular}
\end{bloque-media}
\fi
}



\newcommand{\cajitacontinuacion}[7]{\checkoddpage\ifoddpage
\begin{bloque-media}
\begin{titulo}
\begin{centrador}
	$\!\,$\begin{tabular}{p{32.26\distx}}

	  \\[-3mm]	
	  {\begin{titulo-subseccion-continuacion}  \phantomsection{\section*{#1}} {\Bold\large$\vphantom{\frac{a}{b}}$#1} \end{titulo-subseccion-continuacion}}  \\[-5mm]
	   \\[-1.3pt]
	\end{tabular}
\end{centrador}
\end{titulo}

\begin{tabular}{p{11\distx}p{21\distx}p{5\distx}}
		\begin{descripcion}
			#2
		\end{descripcion} 
	& 
		\begin{columna-central}
			\begin{subtitulo}
				\centering\footnotesize{\Bold #3} \\
				$\qquad\,\,$#4$\qquad\,\,$
			\end{subtitulo}
	
			\begin{grafica}\centering
				#5
			\end{grafica}
	
			\begin{fuente}
				\footnotesize #6 
			\end{fuente}
		\end{columna-central}
	& 
		\begin{columna-marginal}
			\begin{vacio1}
			
			\end{vacio1}
		
			\begin{nota}
				#7
			\end{nota}
		
			\begin{vacio2}
			
			\end{vacio2}
		\end{columna-marginal}
	\\
\end{tabular}
\end{bloque-media}
\else
\begin{bloque-media}
\begin{titulo}
\begin{centrador-par}
	$\!\,$\begin{tabular}{p{32.26\distx}}

	  \\[-3mm]	
	  \begin{titulo-subseccion-continuacion}  \phantomsection{\section*{#1}} {\Bold\large$\vphantom{\frac{a}{b}}$#1} \end{titulo-subseccion-continuacion} \\[-5mm]
	   \\[-1.3pt]
	\end{tabular}
\end{centrador-par}
\end{titulo}

\begin{tabular}{p{5\distx}p{21.36\distx}p{11\distx}}
		\begin{columna-marginal}
			\begin{vacio1}
			
			\end{vacio1}
		
			\begin{nota}
				#7
			\end{nota}
		
			\begin{vacio2}
			
			\end{vacio2}
		\end{columna-marginal}		
	& 
		\begin{columna-central}
			\begin{subtitulo}
				\centering\footnotesize{\Bold #3} \\
				$\qquad\,\,$#4$\qquad\,\,$
			\end{subtitulo}
	
			\begin{grafica}\centering
				#5
			\end{grafica}
	
			\begin{fuente}
				\footnotesize #6 
			\end{fuente}
		\end{columna-central}
	& 
		\begin{descripcion}
			#2
		\end{descripcion} 
	\\
\end{tabular}
\end{bloque-media}
\fi
}



\newcommand{\cajota}[7]{\checkoddpage\ifoddpage
\begin{bloque-una}
\begin{titulo}
\begin{centrador}
	\begin{tabular}{p{2.5\distx}p{28.8\distx}}
	 & \\[-3mm]	
	 & {\begin{titulo-subseccion}\begin{numero-subseccion}\centering\secnumbering\end{numero-subseccion}  \phantomsection{\secnew{#1}}  \end{titulo-subseccion}}  \\[-5mm]
	 &  \\[-1.3pt]
	\end{tabular}
\end{centrador}
\end{titulo}

\begin{tabular}{p{32.95\distx}p{5\distx}}
		\begin{descripcion-una}
			#2
		\end{descripcion-una}  &
\\
		\begin{columna-central-una} \centering
			\begin{subtitulo-una}
				\centering\footnotesize{\Bold #3} \\
				$\qquad\,\,$#4$\qquad\,\,$
			\end{subtitulo-una}
	
			\begin{grafica-una}\centering
				#5
			\end{grafica-una}
	
			\begin{fuente-una}
				\footnotesize #6 
			\end{fuente-una}
		\end{columna-central-una}
	& 
		\begin{columna-marginal-una}
			\begin{vacio1}
			
			\end{vacio1}
		
			\begin{nota-una}
				#7
			\end{nota-una}
		
			\begin{vacio2}
			
			\end{vacio2}
		\end{columna-marginal-una}
	\\
\end{tabular}
\end{bloque-una}
\else
\begin{bloque-una}
\begin{titulo}
\begin{centrador-par}
	\begin{tabular}{p{2.5\distx}p{28.8\distx}}
	 & \\[-3mm]	
	 & {\begin{titulo-subseccion}\begin{numero-subseccion}\centering\secnumbering\end{numero-subseccion}  \phantomsection{\secnew{#1}}  \end{titulo-subseccion}}  \\[-5mm]
	 &  \\[-1.3pt]
	\end{tabular}
\end{centrador-par}
\end{titulo}

$\!\!\!$\begin{tabular}{p{5\distx}p{32.95\distx}}
		  &
			\begin{descripcion-una}
				#2
			\end{descripcion-una}
\\
		\begin{columna-marginal-una}
			\begin{vacio1}
			
			\end{vacio1}
		
			\begin{nota-una}
				#7
			\end{nota-una}
		
			\begin{vacio2}
			
			\end{vacio2}
		\end{columna-marginal-una}
	&
			\begin{columna-central-una} \centering
				\begin{subtitulo-una}
					\centering\footnotesize{\Bold #3} \\
					$\qquad\,\,$#4$\qquad\,\,$
				\end{subtitulo-una}
		
				\begin{grafica-una}\centering
					#5
				\end{grafica-una}
		
				\begin{fuente-una}
					\footnotesize #6 
				\end{fuente-una}
			\end{columna-central-una}
	\\
\end{tabular}
\end{bloque-una}
\fi
}



\begin{document}

%\includepdf{PortadaEnei.pdf}
\clearpage

$\ $
\vspace{14.5cm}

\noindent\begin{tabular}{p{0.9cm}p{6.8cm}}
& 2014.$\,$ Guatemala, Centro América \\
&\Bold Instituto Nacional de Estadística\\[-0.4cm]
&\color{blue!50!black}\url{www.ine.gob.gt}\\[0.9cm]
\end{tabular}\\
\noindent\begin{tabular}{p{0.9cm}p{6.8cm}}
& Está permitida la reproducción parcial o total de los contenidos de este documento con la mención de la fuente. \\[0.5cm]
 
& Este documento fue elaborado empleando  {\Sans R}, Inkscape y {\Logos \XeLaTeX}.\\
\end{tabular} 


\clearpage

$\ $
\vspace{7cm}

\begin{center}
\Bold \LARGE ENCUESTA NACIONAL DE EMPLEO E INGRESOS\\
ENEI 1-2014
\end{center}
\cleardoublepage

\hoja{
$\ $
\vspace{0.0cm}

\begin{center}
{\Bold \LARGE AUTORIDADES}\\[1cm]


{\Bold \large \color{color1!89!black} JUNTA  DIRECTIVA} \\[0.4cm]


{\Bold Ministerio de Economía}\\
Lic. Sergio de la Torre, Titular\\
Lic. Jacobo Rey Sigfrido Lee, Suplente\\[0.4cm]


{\Bold Ministerio de Finanzas}\\
Lic. Dorval Carías, Titular\\
Lic. Edwin Oswaldo Martínez Cameros, Suplente\\[0.4cm]


{\Bold Ministerio de Agricultura, Ganadería y Alimentación}\\
Ing. Elmer López, Titular\\
Ing. Carlos Alfonso Anzueto, Suplente\\[0.4cm]


{\Bold Ministerio de Energía y Minas}\\
Lic. Erick Archila, Titular\\
Licda. Ivanova Ancheta, Suplente\\[0.4cm]


{\Bold Secretaría de Planificación y Programación de la Presidencia}\\
Licda. Ekaterina Arbolievna Parrilla, Titular\\
Licda. Dora Marina Coc, Suplente\\[0.4cm]


{\Bold Banco de Guatemala}\\
Lic. Julio Roberto Suárez, Titular\\
Lic. Sergio Francisco Recinos Rivera, Suplente\\[0.4cm]



{\Bold Universidad de San Carlos de Guatemala}\\
Ing. Murphy Olimpo Paiz, Titular\\
Lic. Oscar René Paniagua Carrera, Suplente\\[0.4cm]


{\Bold Universidades Privadas}\\
Dr. Oscar Guillermo Peláez, Titular\\
Lic. Ariel Rivera Irías, Suplente\\[0.4cm]


{\Bold Comité Coordinador de Asociaciones\\ Agrícolas, Comerciales, Industriales y Financieras}\\
Lic. Juan Raúl Aguilar , Titular\\
Lic. Oscar Augusto Sequeira, Suplente\\[0.8cm]


{\Bold \large \color{color1!89!black} GERENCIA}\\[0.2cm]
Lic. Rubén Narciso, Gerente\\
Lic. Jaime Mejía Salguero, Subgerente Técnico\\
Ing. Orlando Monzón, Subgerente Administrativo Financiero\\


\end{center}
}
\clearpage

$\ $
\vspace{1cm}

\begin{center}
{\Bold \LARGE EQUIPO RESPONSABLE}\\[2cm]

{\Bold \large \color{color1!89!black} REVISIÓN GENERAL}\\[0.2cm]
Rubén Narciso\\[0.8cm]


{\Bold \large \color{color1!89!black} EQUIPO TÉCNICO}\\[0.2cm]
Carlos Enrique Mancia Chúa\\
Carlos Alberto Ortiz Morales\\
Mynor Abel Flores Folgar\\
Luis Fernando Castellanos\\
Cesar Augusto Calderón Barillas\\
Vivian Guzmán Quiroa\\
Nelson Augusto Santa Cruz López\\
Sucely Marleny Donis Bran\\
Patricia Eugenia Hernández García\\
Hugo Elizardo Rivas Portillo\\
Marvin Isaac Reyes López\\
Fabiola Beatriz Ramírez Pinto\\
Hugo Allan García Monterrosa\\
José Carlos Bonilla Aldana\\[0.8cm]

{\Bold \large \color{color1!89!black} DIAGRAMACIÓN Y DISEÑO}\\[0.2cm]
Ligia Morales\\
José Carlos Bonilla Aldana\\[0.8cm]

{\Bold \large \color{color1!89!black} FOTOGRAFÍAS}\\[0.2cm]
Vivian Guzmán Quiroa\\[0.8cm]



\end{center}\setcounter{page}{0}\cleardoublepage


\swapgeometry

$\ $\\[1cm]

\tableofcontents

\cleardoublepage
	\pagestyle{estandar}
	\setcounter{page}{1}
	\setlength{\arrayrulewidth}{1.0pt}





\cleardoublepage


$\ $\\[1cm]
\thispagestyle{empty}
\noindent {\Bold \LARGE Presentación}




$\ $\\

El Instituto Nacional de Estadística (INE), tiene como misión: \comop diseñar y ejecutar la Política Estadística Nacional, para recopilar, producir, analizar y difundir estadísticas confiables, oportunas, transparentes y eficientes\comcl.  Con el fin de cumplir con esta misión, durante el período comprendido del 22 de abril al 22 de mayo de 2014, el INE llevó a cabo el levantamiento de la primera Encuesta Nacional de Empleo e Ingresos de 2014 (ENEI 1-2014), cuyos resultados se complace en presentar en este informe. 

Es sumamente satisfactorio para el INE hacer de conocimiento público que anualmente se llevarán a cabo dos encuestas de empleo e ingresos, las cuales se realizarán con recursos del Estado guatemalteco. No obstante lo anterior, es propicio reconocer el apoyo financiero y técnico que en años anteriores han proporcionado al INE organismos internacionales y, particularmente, agradecer el acompañamiento técnico que en esta oportunidad brindó la Organización Internacional del Trabajo (OIT) y la Comisión Económica para América Latina y el Caribe (CEPAL).

La ENEI 1-2014 tiene como principal objetivo obtener información estadística sobre variables de empleo, desempleo, subempleo y actividad e inactividad económica de la población. En lo que respecta a los ingresos, se captó información sobre formas, fuentes, montos, beneficios sociales y laborales, ayudas en especie y dinero, e ingresos no laborales. Complementariamente, se obtuvo información sociodemográfica como sexo, edad, estado conyugal, pertenencia a pueblos y educación. 

Se espera que la información que se incluye en el presente documento contribuya tanto al análisis de las características del mercado laboral guatemalteco, como a la formulación de políticas públicas en materia laboral, que beneficien a la población guatemalteca.

Resta únicamente agradecer a los ciudadanos la información proporcionada para la ENEI 1-2014, la cual se enmarca en el Sistema Integrado de Encuestas de Hogares (SIEH).

$\ $

\noindent Atentamente,


\begin{center}
Rubén Darío Narciso Cruz\\
Gerente\\
Instituto Nacional de Estadística
\end{center}


\cleardoublepage


$\ $\\[0cm]

\noindent {\Bold \LARGE Antecendentes y objetivos de la encuesta}\\[7mm]



\noindent {\color{color1!80!black}\Bold\Large Antecedentes}\\[1mm] \thispagestyle {empty}

De2002 a la fecha, el Instituto Nacional de Estadística (INE) viene realizando esfuerzos para generar información estadística que posibilite estudiar y analizar los diferentes fenómenos asociados al mercado de trabajo, y que sirvan de insumo para la instrumentación de programas tendentes a mejorar las condiciones laborales de la población guatemalteca. 

En efecto, entre 2002 y 2003 el INE llevó a cabo cuatro encuestas de empleo, con carácter trimestral, las cuales permitieron conocer la estructura y evolución del mercado de trabajo. Estas encuestas cubrieron tres dominios de estudio: urbano metropolitano (dominio 1); resto urbano (dominio 2), integrado por las áreas urbanas de todos los departamentos de la República, exceptuando el departamento de Guatemala; y rural nacional (dominio 3), representativo de las áreas rurales del país.

Entre septiembre y noviembre de 2004, el INE realizó una encuesta utilizando un mayor tamaño de muestra, la que generó información a nivel departamental urbano y rural nacional. Posteriormente, en 2009, con el apoyo financiero del Banco Interamericano de Desarrollo (BID), se programaron tres encuestas adicionales, las que se llevaron a cabo en2010, 2011 y 2012.

Para 2013, nuevamente se planificó la realización de dos encuestas de empleo e ingresos, con recursos provenientes del Presupuesto Nacional. La primera de ellas (ENEI 1-2013), se realizó entre abril y mayo; la segunda (ENEI 2-2013),durante los meses de octubre y noviembre.

Al igual que en anteriores encuestas de empleo e ingreso, la ENEI 1-2014 fue diseñada para satisfacer la necesidad de caracterizar las variables relacionadas con los temas de actividad económica y social de la población guatemalteca; así como generar información actualizada sobre las tendencias de los indicadores de empleo, desempleo y subempleo, requeridas por el sector público y privado, así como por organismos internacionales, para evaluar técnicamente la actividad socioeconómica del país y diseñar e implementar políticas y programas en el ámbito laboral.\\[6mm]


\noindent{\color{color1!80!black}\Bold\Large Objetivos de la ENEI 1-2014}\\[1mm] 


\noindent\textbf{General:}

Fortalecer las estadísticas que explican la evolución y caracterización del mercado de trabajo y que permiten analizar más ampliamente los temas relativos a la actividad económica en general, así como el Sistema Integrado de Encuestas de Hogares (SIEH).


\noindent\textbf{Específicos:}\\[-0.5cm]


\begin{itemize}
\item[A.] Producir información sobre:

\begin{itemize}
\item[$\bullet$] Las diferentes variables del empleo; el desempleo; la actividad e inactividad económica de la población; y la inserción laboral, incluida la caracterización social y económica de las personas involucradas.
\item[$\bullet$] Las características, composición, estructura y funcionamiento del mercado de trabajo.
\end{itemize}

\item[B.]	Contar con indicadores relacionados con el mercado de trabajo, que permitan analizar su coyuntura y tendencias dentro de la actividad económica y social del país.

\item[C.]	Generar información sobre las principales variables del contexto socioeconómico y familiar que inciden en el mercado de trabajo, como la seguridad social, la educación, las condiciones de habitación, la composición familiar y la generación de formas colectivas de ingresos al interior de los hogares.
\end{itemize}\thispagestyle{empty}\setcounter{page}{0}
\cleardoublepage


\restoregeometry




%--------------------------CONTENIDOS---------------------------




%--------------------inicio de capítulo 1-------------------------------------------

\INEchapter{Contexto macroeconómico y demográfico}{Contexto macroeconómico }{y demográfico}{\quad Las características de la población (en términos de edades, sexo, escolaridad; entre otras) determinan las cualidades y cómo se compone  la fuerza laboral de un país. \\[3mm]
	
\quad Por otro lado, la actividad económica y el modelo de desarrollo determinan grandemente la demanda en el mercado laboral, es decir, la capacidad de la economía de generar puestos de trabajo y las características de estos.
}




\hoja{
	\cajita{Producto Interno Bruto}{Hola esto es una descripción}{Crecímiento anual del PIB a precios constantes de 2001Crecimiento anual del PIB a precios constantes de 20}{ Crecimiento anual del PIB a precios constantes de 20}{\ \\[-4mm]\begin{tikzpicture}[x=1pt,y=1pt]
		\input{prueba}
		\end{tikzpicture}} {Fuente: Banco de Guatemala \guiop BANGUAT \guicl)}{ }
 	
		\cajita{Producto Interno Bruto}{Hola esto es una descripción}{Crecimiento anual del PIB a precios constantes de 2001Crecimiento anual del PIB a precios constantes de 20}{ Crecimiento anual del PIB a precios constantes de 20}{\ \\[-4mm]\begin{tikzpicture}[x=1pt,y=1pt]
			\input{prueba2}
			\end{tikzpicture}} {Fuente: Banco de Guatemala \guiop BANGUAT \guicl)}{ }

}

\hoja{
	\cajita{Producto Interno Bruto}{Hola esto es una descripción}{Crecímiento anual del PIB a precios constantes de 2001Crecimiento anual del PIB a precios constantes de 20}{ Crecimiento anual del PIB a precios constantes de 20}{\ \\[-4mm]\begin{tikzpicture}[x=1pt,y=1pt]
		\input{prueba007}
		\end{tikzpicture}} {Fuente: Banco de Guatemala \guiop BANGUAT \guicl)}{ }
	
	\cajita{Producto Interno Bruto}{Hola esto es una descripción}{Crecimiento anual del PIB a precios constantes de 2001Crecimiento anual del PIB a precios constantes de 20}{ Crecimiento anual del PIB a precios constantes de 20}{\ \\[-4mm]\begin{tikzpicture}[x=1pt,y=1pt]
		\input{prueba107}
		\end{tikzpicture}} {Fuente: Banco de Guatemala \guiop BANGUAT \guicl)}{ }
	
}



\hoja{
	\cajita{Producto Interno Bruto}{Hola esto es una descripción}{Crecímiento anual del PIB a precios constantes de 2001Crecimiento anual del PIB a precios constantes de 20}{ Crecimiento anual del PIB a precios constantes de 20}{\ \\[-4mm]\begin{tikzpicture}[x=1pt,y=1pt]
		\input{prueba009}
		\end{tikzpicture}} {Fuente: Banco de Guatemala \guiop BANGUAT \guicl)}{ }
	
	\cajita{Producto Interno Bruto}{Hola esto es una descripción}{Crecimiento anual del PIB a precios constantes de 2001Crecimiento anual del PIB a precios constantes de 20}{ Crecimiento anual del PIB a precios constantes de 20}{\ \\[-4mm]\begin{tikzpicture}[x=1pt,y=1pt]
		\input{prueba110}
		\end{tikzpicture}} {Fuente: Banco de Guatemala \guiop BANGUAT \guicl)}{ }
	
}

\hoja{
	\cajita{Producto Interno Bruto}{Hola esto es una descripción}{Crecímiento anual del PIB a precios constantes de 2001Crecimiento anual del PIB a precios constantes de 20}{ Crecimiento anual del PIB a precios constantes de 20}{\ \\[-4mm]\begin{tikzpicture}[x=1pt,y=1pt]
		\input{prueba210}
		\end{tikzpicture}} {Fuente: Banco de Guatemala \guiop BANGUAT \guicl)}{ }
	
	\cajita{Producto Interno Bruto}{Hola esto es una descripción}{Crecimiento anual del PIB a precios constantes de 2001Crecimiento anual del PIB a precios constantes de 20}{ Crecimiento anual del PIB a precios constantes de 20}{\ \\[-4mm]\begin{tikzpicture}[x=1pt,y=1pt]
		\input{prueba110}
		\end{tikzpicture}} {Fuente: Banco de Guatemala \guiop BANGUAT \guicl)}{ }
	
}

\hoja{
	\cajita{Producto Interno Bruto}{Hola esto es una descripción}{Crecímiento anual del PIB a precios constantes de 2001Crecimiento anual del PIB a precios constantes de 20}{ Crecimiento anual del PIB a precios constantes de 20}{\ \\[-4mm]\begin{tikzpicture}[x=1pt,y=1pt]
		\input{prueba310}
		\end{tikzpicture}} {Fuente: Banco de Guatemala \guiop BANGUAT \guicl)}{ }
	
	\cajita{Producto Interno Bruto}{Hola esto es una descripción}{Crecimiento anual del PIB a precios constantes de 2001Crecimiento anual del PIB a precios constantes de 20}{ Crecimiento anual del PIB a precios constantes de 20}{\ \\[-4mm]\begin{tikzpicture}[x=1pt,y=1pt]
		\input{prueba110}
		\end{tikzpicture}} {Fuente: Banco de Guatemala \guiop BANGUAT \guicl)}{ }
	
}
% -------------------------Fin de hoja-----------------------------






%\includepdf{ContraportadaEnei.pdf}
\end{document}